%\title{My two column CV}
%
% tccv (two columns curriculum vitae) is a LaTeX class inspired by
% the template found at latextemplates.com by Alessandro Plasmati.
%
% Create by Nicola Fontana, the original files can be downloaded from:
% http://dev.entidi.com/p/tccv/
%
\documentclass{tccv}
\usepackage[english]{babel}
\usepackage{biblatex}
\usepackage{csquotes}
\usepackage[ddmmyyyy]{datetime}

\addbibresource{references.bib}

\newcommand{\sphere}{360\textdegree~}

\newcommand{\updateinfo}[1][\today]{\par\vfill\hfill{\scriptsize\color{darkergray}Last updated on #1}}

\begin{document}

{\usekomafont{part} Eugene Lyapustin} \hfill \faMapMarker~Yerevan, Armenia \\
{\Large Research Engineer, Computer Vision} \hfill \faPhone~+7 925 538 39 58 \\
Open to relocation \hfill \faEnvelope~\href{mailto:unishifft@gmail.com}{unishifft@gmail.com} ~ \faTelegramPlane~\href{https://t.me/unishifft}{unishifft}\\
\null\hfill \faLinkedin~\href{https://linkedin.com/in/eugene-lyapustin}{eugene-lyapustin} ~ \faGithub~\href{https://github.com/unishift/}{unishift}

% \part{Eugene Lyapustin}{Date of Birth: 06.03.1998 ~ Moscow, Russia ~ \href{mailto:unishifft@gmail.com}{unishifft@gmail.com}}

\section{Summary}

Creative research engineer with 4 years of academic experience and contributions to large code bases both open-source (FFmpeg, PyTorch) and corporate (VK). Advanced in deep learning, 2D/3D computer vision, image/video quality assessment and video processing.

\section{Education}

\begin{eventlist}

\evententry{}
     {Lomonosov Moscow State University}
     {Master of Computer Science, GPA: 4.72/5.00}
     {2020 -- 2022}
\begin{itemize}
     \item Faculty of Computational Mathematics and Cybernetics, Graphics \& Media Lab
     \item Completed courses on:\\
          \faAngleRight~ Computer Graphics\qquad
          \faAngleRight~ Machine Learning\qquad
          \faAngleRight~ Deep Learning\qquad
          \faAngleRight~ Computer Vision
     \item Specialization: Computer Vision, Graphics and Image Processing
     \item Research on Video Super-Resolution
     % \item Taught media compression and video processing
     % \item Supervised undergraduate students
\end{itemize}

% \evententry{}
%      {}
%      {Bachelor of Computer Science, GPA: 4.70/5.00}
%      {2016 -- 2020}
% \begin{itemize}
%           \item Research on \sphere video processing 
% \end{itemize}

\end{eventlist}

\section{Work experience}

\begin{eventlist}

\evententry{October 2021 -- Present}
     {VK}
     {Research Engineer}
     {Moscow, Russia}
\begin{itemize}
     \item Currently developing multi-modal ranking models for video search engine
     \item Developed and integrated computer vision models for video classification of NSFW, spam, duplicates
     \item Set crowdsource dataset collection for models training
\end{itemize}

\evententry{Sep 2018 -- Aug 2022}
     {MSU Graphics \& Media Lab}
     {Research \& Development Engineer}
     {Moscow, Russia}
\begin{itemize}
     \item Developed detail restoration assessment method for Video Super-Resolution using deep learning 
     \item Developed quality metric for video-conference applications using machine learning methods
     \item Developed \sphere video stabilization algorithm using geometric vision and computational geometry
     \item Supervised research team of 5 students on Video Super-Resolution project \href{https://videoprocessing.ai/benchmarks/video-super-resolution.html}{~\faExternalLink*}
\end{itemize}

\evententry{May 2021 -- Aug 2021}
     {Google Summer of Code @ Intel® Video and Audio for Linux}
     {Student Developer}
     {Remote}
\begin{itemize}
     \item Implemented \sphere video projection conversion for libXcam library\hfill
     \href{https://unishift.github.io/project/2021-08-20-gsoc-libxcam}{Project page~\faExternalLink*}
     \item Used GLES for GPU-acceleration
\end{itemize}

\evententry{May 2019 -- Aug 2019}
     {Google Summer of Code @ FFmpeg}
     {Student Developer}
     {Remote}
\begin{itemize}
     \item Developed \texttt{v360} filter for \sphere video projection conversion for ffmpeg\hfill
     \href{https://unishift.github.io/project/2019-08-26-gsoc-ffmpeg}{Project page~\faExternalLink*}
\end{itemize}

\end{eventlist}

\section{Technical stack}

\begin{factlist}

\item{Python}
     {NumPy, SciPy, scikit-learn, Pandas, OpenCV, PyTorch, TensorFlow, ONNX}

\item{C++}
     {OpenCV, Eigen, Ceres Solver, CUDA, OpenCL}

\item{Java}
     {Hadoop}

\item{other}
     {Linux, Git, CI, \LaTeX}

\end{factlist}


\section{Languages}

\faAngleDoubleRight~ \textsc{English}\enspace{Full working proficiency}\qquad
\faAngleDoubleRight~ \textsc{Russian}\enspace{Native speaker}

\section{Publications}

\begin{factlist}
     \item{2022}{ERQA: Edge-restoration Quality Assessment for Video Super-Resolution, VISAPP [\href{https://arxiv.org/abs/2110.09992}{Paper}] [\href{https://github.com/msu-video-group/ERQA}{Code}]}
     % \item{2022}{Towards True Detail Restoration for Super-Resolution: A Benchmark and a Quality Metric, under revision [\href{https://arxiv.org/abs/2203.08923}{Paper}]}
\end{factlist}

\updateinfo

\end{document}
