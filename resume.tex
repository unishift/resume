%\title{My two column CV}
%
% tccv (two columns curriculum vitae) is a LaTeX class inspired by
% the template found at latextemplates.com by Alessandro Plasmati.
%
% Create by Nicola Fontana, the original files can be downloaded from:
% http://dev.entidi.com/p/tccv/
%
\documentclass{tccv}
\usepackage[english]{babel}
\usepackage{biblatex}
\usepackage{csquotes}
\usepackage[ddmmyyyy]{datetime}

\addbibresource{references.bib}

\newcommand{\sphere}{360\textdegree~}

\newcommand{\updateinfo}[1][\today]{\par\vfill\hfill{\scriptsize\color{darkergray}Last updated on #1}}

\begin{document}

{\usekomafont{part} Eugene Lyapustin} \hfill \faMapMarker~Yerevan, Armenia \\
{\Large Research Engineer, Computer Vision} \hfill \faPhone~+7 925 538 39 58 \\
\null \hfill \faEnvelope~\href{mailto:unishifft@gmail.com}{unishifft@gmail.com} ~ \faTelegramPlane~\href{https://t.me/unishifft}{unishifft}\\
\null\hfill \faLinkedin~\href{https://linkedin.com/in/eugene-lyapustin}{eugene-lyapustin} ~ \faGithub~\href{https://github.com/unishift/}{unishift}

% \section{}

% Creative research engineer with 4 years of academic experience and contributions to large code bases both open-source (FFmpeg, PyTorch) and corporate (VK). Advanced in deep learning, 2D/3D computer vision, image/video quality assessment, and video processing.

\section{Work experience}

\begin{eventlist}

\evententry{October 2021 -- Present}
     {VK}
     {Research Engineer}
     {Moscow, Russia}
\begin{itemize}
     \item Developed multi-modal ranking models for video search engine \\ using PyTorch and CLIP architecture to calculate similarity between text query and video
     \item Developed computer vision models for video classification of NSFW, spam, and near duplicates \\ using PyTorch and transfer learning to small datasets
     \item Implemented scripts for automatic video processing such as frame extraction and resizing \\ using Python and FFmpeg
     \item Optimized computer vision models for production inference by converting them \\ to ONNX format and quantizing to lower precision with 5 times speed improvement
\end{itemize}

\evententry{Sep 2018 -- Aug 2022}
     {MSU Graphics \& Media Lab}
     {Research \& Development Engineer}
     {Moscow, Russia}
\begin{itemize}
     \item Developed detail restoration assessment method for Video Super-Resolution \\ using PyTorch and segmentation models
     \item Developed quality metric for video-conference applications using sklearn \\ and regression models with 20\% higher correlation with subjective evaluation 
     \item Developed \sphere video stabilization algorithm using C++ and OpenCV by applying \\ geometric vision for 3D camera path reconstruction
     \item Supervised research team of 5 students on Video Super-Resolution project \href{https://videoprocessing.ai/benchmarks/video-super-resolution.html}{~\faExternalLink*}
\end{itemize}

\evententry{May 2021 -- Aug 2021}
     {Google Summer of Code @ Intel® Video and Audio for Linux}
     {Student Developer}
     {Remote}
\begin{itemize}
     \item Implemented \sphere video projection conversion for libXcam library\hfill
     \href{https://unishift.github.io/project/2021-08-20-gsoc-libxcam}{Project page~\faExternalLink*}
\end{itemize}

\evententry{May 2019 -- Aug 2019}
     {Google Summer of Code @ FFmpeg}
     {Student Developer}
     {Remote}
\begin{itemize}
     \item Developed \texttt{v360} filter for \sphere video projection conversion for FFmpeg\hfill
     \href{https://unishift.github.io/project/2019-08-26-gsoc-ffmpeg}{Project page~\faExternalLink*}
     \item Used computational geometry for applying transformations between sphere projections
\end{itemize}

\end{eventlist}

\section{Education}

\begin{eventlist}

\evententry{}
     {Lomonosov Moscow State University}
     {Master of Computer Science, GPA: 4.72/5.00}
     {2020 -- 2022}
     \begin{itemize}
          \item Thesis: A method for details restoration assessment of Video Super-Resolution
     \end{itemize}
% \begin{itemize}
     % \item Faculty of Computational Mathematics and Cybernetics, Graphics \& Media Lab
     % \item Completed courses on:\\
     %      \faAngleRight~ Computer Graphics\qquad
     %      \faAngleRight~ Machine Learning\qquad
     %      \faAngleRight~ Deep Learning\qquad
     %      \faAngleRight~ Computer Vision
     % \item Specialization: Computer Vision, Graphics and Image Processing
     % \item Research on Video Super-Resolution
     % \item Taught media compression and video processing
     % \item Supervised undergraduate students
% \end{itemize}

\evententry{}
     {}
     {Bachelor of Computer Science, GPA: 4.70/5.00}
     {2016 -- 2020}
     \begin{itemize}
          \item Thesis: A method for \sphere video stabilization and enhancement
     \end{itemize}

\end{eventlist}

\section{Skills}

\begin{factlist}

\item{Industry Knowledge}
     {Computer Vision, Machine Learning, Deep Learning, Geometric Vision, Video Quality Assessment}

\item{Technical Skills}
     {Python, PyTorch, Pandas, matplotlib, ONNX, C/C++, OpenCV, Eigen, Ceres Solver, OpenCL, Java, Hadoop}

\item{Soft Skills}
     {Problem Solving, Teamwork, Presentations, Teaching, Leadership}

\end{factlist}

\section{Languages}

\faAngleDoubleRight~ \textsc{English}\enspace{Full working proficiency}\qquad
\faAngleDoubleRight~ \textsc{Russian}\enspace{Native speaker}

\section{Publications}

\begin{factlist}
     \item{2022}{ERQA: Edge-restoration Quality Assessment for Video Super-Resolution, VISAPP [\href{https://arxiv.org/abs/2110.09992}{Paper}] [\href{https://github.com/msu-video-group/ERQA}{Code}]}
     % \item{2022}{Towards True Detail Restoration for Super-Resolution: A Benchmark and a Quality Metric, under revision [\href{https://arxiv.org/abs/2203.08923}{Paper}]}
\end{factlist}

\updateinfo

\end{document}
